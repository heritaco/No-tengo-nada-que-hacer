\documentclass[8pt]{extarticle} % Tamaño de fuente base 12pt
\usepackage{hyperref} % Para enlaces
\usepackage{enumitem} % Para listas personalizadas
\usepackage[utf8]{inputenc} % Codificación UTF-8 (necesario en algunos sistemas)
\usepackage[T1]{fontenc} % Codificación de fuente T1
\usepackage[english]{babel} % Lenguaje en inglés
\usepackage[letterpaper, left=2.5cm, top=1cm, right=2.5cm, bottom=0.49cm]{geometry} % Márgenes personalizados
\usepackage{helvet} % Fuente Helvetica
\renewcommand{\familydefault}{\sfdefault} % Usa Helvetica como fuente por defecto
\usepackage{setspace} % Paquete para interlineado
\usepackage{lipsum} % Para texto de ejemplo
\usepackage{xcolor}


\usepackage{multicol}
\usepackage{parskip} % Espaciado entre párrafos
\usepackage{ragged2e} % Justificación completa

% Configurar interlineado
\setstretch{1.2} % Establece el interlineado a 1.2 (más espacioso)

% Configuración de párrafos
\setlength{\parindent}{0pt} % Sin sangría al inicio de los párrafos





\definecolor{customblue}{HTML}{001c7f}




\begin{document}

% Cambia el tamaño de la fuente a 10pt antes de tu nombre  
{\fontsize{10pt}{12pt}\selectfont  
\begin{center}  
    \textbf{Heriberto Espino Montelongo}\\  
    \vspace{1ex} % Reducción de espacio  
\end{center}  
}

\begin{center}  
    Puebla, México, 72160 \textbullet \   
    \href{mailto:heriberto.espinomo@udlap.mx}{\textcolor{customblue}{\underline{heriberto.espinomo@udlap.mx}}} \textbullet \   
    \href{tel:+522228101202}{\textcolor{customblue}{\underline{+52 222 810 1202}}} \textbullet \   
    \href{https://github.com/heritaco}{\textcolor{customblue}{\underline{GitHub: heritaco}}}  
\end{center}

\begin{center}  
    \vspace{5ex}  
    \textbf{Formación Académica}  
    \vspace{-2ex}  
\end{center}

\textbf{Universidad de las Américas Puebla} \hfill Puebla, México\\  
Licenciatura en Actuaría, promedio: 9.5/10. \hfill Agosto 2026 \\  
Cursos relevantes: Portafolios de Inversión, Modelos Estocásticos, Análisis de Regresión, Análisis de Series Temporales, Teoría del Riesgo, Teoría de la Medida, Demografía.\\   

\textbf{Universidad de las Américas Puebla} \hfill Puebla, México\\  
Licenciatura en Ciencia de Datos, promedio: 9.6/10. \hfill Agosto 2026 \\  
Cursos relevantes: Optimización Avanzada, Reconocimiento de Patrones, Inteligencia Artificial, Minería de Datos, Análisis Topológico de Datos, Análisis Geoespacial de Datos.





\begin{center}  
    \vspace{2ex}  
    \textbf{Experiencia Profesional}  
    \vspace{-1ex}  
\end{center}

\textbf{Actinver} \hfill Ciudad de México, México\\  
\textit{Practicante} \hfill Junio 2025 - Julio 2025\\  
Lorem Ipsum. Lorem Ipsum. Lorem Ipsum. Lorem Ipsum. Lorem Ipsum. Lorem Ipsum. Lorem Ipsum. Lorem Ipsum. Lorem Ipsum. Lorem Ipsum. Lorem Ipsum. Lorem Ipsum. Lorem Ipsum. Lorem Ipsum. Lorem Ipsum. Lorem Ipsum. Lorem Ipsum. Lorem Ipsum.\\

\textbf{Desarrollador de Herramientas para Series de Tiempo — Colaboración Académica} \hfill Dra. Daniela Cortés Toto \\
\textit{Practicante} \hfill Mayo 2025\\  
Desarrollo de librerías en Python enfocadas en el modelado y simulación de series de tiempo financieras, en colaboración con profesores del departamento de Matemáticas y Finanzas Cuantitativas. \\

\textbf{Desarrollador de Herramientas para Series de Tiempo — Colaboración Académica} \hfill Hector Saib Maravillo \\
\textit{Practicante} \hfill Mayo 2025\\  
Desarrollo de librerías en Python enfocadas en el modelado y simulación de series de tiempo financieras, en colaboración con profesores del departamento de Matemáticas y Finanzas Cuantitativas. \\




\begin{center}  
    \vspace{1ex}  
    \textbf{Actividades}  
    \vspace{-1ex}  
\end{center}

\textbf{Olimpiada Mexicana Universitaria de Matemáticas} \hfill México\\  
\textit{Competidor} \hfill Junio 2025\\  
Lorem Ipsum. Lorem Ipsum. Lorem Ipsum. Lorem Ipsum. Lorem Ipsum. Lorem Ipsum. Lorem Ipsum. Lorem Ipsum. Lorem Ipsum. Lorem Ipsum. Lorem Ipsum. Lorem Ipsum. Lorem Ipsum. Lorem Ipsum. Lorem Ipsum. Lorem Ipsum. Lorem Ipsum. Lorem Ipsum.\\

\textbf{The William Lowell Putnam Matemathical Competition} \hfill Norteamérica\\  
\textit{Competidor} \hfill Diciembre 2024, Diciembre 2025\\  
Participación en una de las competencias universitarias de matemáticas más prestigiosas de Norteamérica. Resolución problemas complejos en teoría de números, álgebra, combinatoria, geometría euclideana, cálculo, entre otros.\\

\textbf{Olimpiada de Matemáticas de la Universidad de las Américas Puebla} \hfill Puebla, México\\  
\textit{Participante} \hfill Octubre 2024, Abril 2025\\  
Clasificación dentro de los 10 mejores entre estudiantes de diferentes licenciaturas. Resolución de problemas en álgebra, álgebra lineal, combinatoria, estadística, geometría analítica, teoría de conjuntos y cálculo.\\

\textbf{Reto Actinver} \hfill México\\  
\textit{Participante} \hfill Octubre 2024, Octubre 2025\\
Reto financiero nacional enfocado en estrategias de inversión. Se realizó un análisis cuantitativo de datos de mercado para 201 activos, empleando la frontera eficiente de Markowitz para construir portafolios óptimos y se seleccionó el portafolio con mayor ratio de Sharpe en un entorno simulado de inversión.\\

\textbf{Reto Coppel} \hfill Puebla, México\\  
\textit{Participante} \hfill Marzo 2025\\  
Competencia universitaria enfocada en la optimización del tiempo de espera en sucursales de Coppel. Se identificaron distribuciones de tiempo de espera por clientes atendidos y no atendidos, se crearon variables y se aplicó clustering espectral para clasificar las tiendas en categorías de bajo, medio y alto tiempo de espera. \\








\newpage


\begin{center}
\vspace{1ex}
\textbf{Organizaciones Estudiantiles}
\vspace{-1ex}
\end{center}

\textbf{Bitwise Competitive Programming} \hfill Universidad de las Américas Puebla\\
\textit{Miembro} \hfill Enero 2025 - Diciembre 2025\\
Participación en competencias de programación algorítmica bajo presión de tiempo, con enfoque en la optimización de complejidad computacional. Desarrollo de soluciones eficientes en C++ y Python para problemas de grafos, estructuras de datos y programación dinámica. \\

\textbf{How Do Machines Learn?} \hfill Universidad de las Américas Puebla\\
\textit{Miembro} \hfill Enero 2025 - Diciembre 2025\\
Exploración aplicada del aprendizaje profundo utilizando librerías como PyTorch y CUDA. Implementación de redes neuronales desde cero y entrenamiento de modelos en GPU. Estudio de artículos y arquitecturas como CNN y RNN. \\

\textbf{Hackztecks} \hfill Universidad de las Américas Puebla\\
\textit{Miembro} \hfill Enero 2025 - Diciembre 2025 \\
Lorem Ipsum. Lorem Ipsum. Lorem Ipsum. Lorem Ipsum. Lorem Ipsum. Lorem Ipsum. Lorem Ipsum. Lorem Ipsum. Lorem Ipsum. Lorem Ipsum. Lorem Ipsum. Lorem Ipsum. Lorem Ipsum. Lorem Ipsum. Lorem Ipsum. Lorem Ipsum. Lorem Ipsum. Lorem Ipsum.\\

\textbf{Phi Psi Kappa} \hfill Universidad de las Américas Puebla\\
\textit{Miembro} \hfill Enero 2025 - Diciembre 2025 \\
Lorem Ipsum. Lorem Ipsum. Lorem Ipsum. Lorem Ipsum. Lorem Ipsum. Lorem Ipsum. Lorem Ipsum. Lorem Ipsum. Lorem Ipsum. Lorem Ipsum. Lorem Ipsum. Lorem Ipsum. Lorem Ipsum. Lorem Ipsum. Lorem Ipsum. Lorem Ipsum. Lorem Ipsum. Lorem Ipsum.\\



\begin{center}
\vspace{1ex}
\textbf{Certificaciones}
\vspace{-1ex}
\end{center}

\textbf{Computo Cuántico} \hfill Universidad Nacional Autónoma de México\\
\textit{Miembro} \hfill Junio 2025\\
Lorem Ipsum. Lorem Ipsum. Lorem Ipsum. Lorem Ipsum. Lorem Ipsum. Lorem Ipsum. Lorem Ipsum. Lorem Ipsum. Lorem Ipsum. Lorem Ipsum. Lorem Ipsum. Lorem Ipsum. Lorem Ipsum. Lorem Ipsum. Lorem Ipsum. Lorem Ipsum. Lorem Ipsum. Lorem Ipsum.\\

\textbf{How Do Machines Learn?} \hfill Universidad Nacional Autónoma de México \\
\textit{Miembro} \hfill Enero 2025 - Diciembre 2025\\
Lorem Ipsum. Lorem Ipsum. Lorem Ipsum. Lorem Ipsum. Lorem Ipsum. Lorem Ipsum. Lorem Ipsum. Lorem Ipsum. Lorem Ipsum. Lorem Ipsum. Lorem Ipsum. Lorem Ipsum. Lorem Ipsum. Lorem Ipsum. Lorem Ipsum. Lorem Ipsum. Lorem Ipsum. Lorem Ipsum. \\


\begin{center}  
    \vspace{1ex}  
    \textbf{Habilidades} 
    \vspace{-1ex}  
\end{center}


\begin{multicols}{2}
\textbf{Lenguajes de Programación:} Python, R, C++, SQL, Java, Mosel, C.

\textbf{Herramientas de Office:} Excel (VBA), Word, PowerPoint, PowerBI.

\textbf{Sistemas de Gestión de Bases de Datos:} MySQL.

\textbf{Control de Versiones:} Git, GitHub.

\textbf{Entornos de Scripting y Consola:} Bash (Arch Linux, Debian), PowerShell (Windows).

\textbf{Lenguajes de Marcado y Documentación:} HTML, LaTeX, Markdown.
\end{multicols}


1 agosto 2021 

2 enero 2022


3 agosto 2022
4 enero 2023
5 agosto 2023

6 enero 2024

7 agosto 2024

8 enero 2025


9 agosto 2025
10 enero 2026

\end{document}









