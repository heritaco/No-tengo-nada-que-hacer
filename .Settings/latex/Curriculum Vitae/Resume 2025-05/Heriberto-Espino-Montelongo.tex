\documentclass[8pt]{extarticle} % Tamaño de fuente base 12pt
\usepackage{hyperref} % Para enlaces
\usepackage{enumitem} % Para listas personalizadas
\usepackage[utf8]{inputenc} % Codificación UTF-8 (necesario en algunos sistemas)
\usepackage[T1]{fontenc} % Codificación de fuente T1
\usepackage[english]{babel} % Lenguaje en inglés
\usepackage[letterpaper, left=2.5cm, top=1.5cm, right=2.5cm, bottom=0.49cm]{geometry} % Márgenes personalizados
\usepackage{helvet} % Fuente Helvetica
\renewcommand{\familydefault}{\sfdefault} % Usa Helvetica como fuente por defecto
\usepackage{setspace} % Paquete para interlineado
\usepackage{lipsum} % Para texto de ejemplo
\usepackage{xcolor}

% Configurar interlineado
\setstretch{1.2} % Establece el interlineado a 1.2 (más espacioso)

% Configuración de párrafos
\setlength{\parindent}{0pt} % Sin sangría al inicio de los párrafos





\definecolor{customblue}{HTML}{001c7f}




\begin{document}

% Cambia el tamaño de la fuente a 10pt antes de tu nombre  
{\fontsize{10pt}{12pt}\selectfont  
\begin{center}  
    \textbf{Heriberto Espino Montelongo}\\  
    \vspace{-1ex} % Reducción de espacio  
\end{center}  
}

\begin{center}  
    Puebla, México, 72160 \textbullet \   
    \href{mailto:heriberto.espinomo@udlap.mx}{\textcolor{customblue}{\underline{heriberto.espinomo@udlap.mx}}} \textbullet \   
    \href{tel:+522228101202}{\textcolor{customblue}{\underline{222 810 1202}}} \textbullet \   
    \href{https://github.com/heritaco}{\textcolor{customblue}{\underline{GitHub: heritaco}}}  
\end{center}

\begin{center}  
    \vspace{5ex}  
    \textbf{Formación Académica}  
    \vspace{-2ex}  
\end{center}

\textbf{Universidad de las Américas Puebla} \hfill Puebla, México\\  
Licenciatura en Ciencia Actuarial, promedio: 9.5/10. \hfill 42 de 50 cursos completados (2021 - Presente) \\  
Cursos relevantes: Portafolios de Inversión, Modelos Estocásticos, Análisis de Regresión, Análisis de Series Temporales, Teoría del Riesgo, Teoría de la Medida, Demografía.\\  
\\  
\textbf{Universidad de las Américas Puebla} \hfill Puebla, México\\  
Licenciatura en Ciencia de Datos, promedio: 9.6/10. \hfill 42 de 50 cursos completados (2021 - Presente)\\  
Cursos relevantes: Optimización Avanzada, Reconocimiento de Patrones, Inteligencia Artificial, Minería de Datos, Análisis Topológico de Datos, Análisis Geoespacial de Datos.

\begin{center}  
    \vspace{1ex}  
    \textbf{Actividades}  
    \vspace{-1ex}  
\end{center}

\textbf{The William Lowell Putnam Matemathical Competition} \hfill Norteamérica\\  
\textit{Competidor} \hfill Diciembre 2024\\  
Participación en una de las competencias universitarias de matemáticas más prestigiosas de Norteamérica. Resolución problemas complejos en teoría de números, álgebra, combinatoria, geometría euclideana, cálculo, entre otros.\\

\textbf{Olimpiada de Matemáticas UDLAP} \hfill Puebla, México\\  
\textit{Participante} \hfill Agosto 2024,  Febrero 2025\\  
Clasificación dentro de los 10 mejores entre estudiantes de diferentes licenciaturas. Resolución de problemas en álgebra, álgebra lineal, combinatoria, estadística, geometría analítica, teoría de conjuntos y cálculo.\\

\textbf{Reto Actinver 2024} \hfill México\\  
\textit{Participante} \hfill Septiembre 2024\\  
Reto financiero nacional enfocado en estrategias de inversión. Se realizó un análisis cuantitativo de datos de mercado para 201 activos, empleando la frontera eficiente de Markowitz para construir portafolios óptimos y se seleccionó el portafolio con mayor ratio de Sharpe en un entorno simulado de inversión.\\

\textbf{Reto Coppel} \hfill Puebla, México\\  
\textit{Participante} \hfill Marzo 2025\\  
Competencia universitaria enfocada en la optimización del tiempo de espera en sucursales de Coppel. Se identificaron distribuciones de tiempo de espera por clientes atendidos y no atendidos, se crearon variables y se aplicó clustering espectral para clasificar las tiendas en categorías de bajo, medio y alto tiempo de espera. \\

\begin{center}
\vspace{1ex}
\textbf{Organizaciones Estudiantiles}
\vspace{-1ex}
\end{center}

\textbf{Bitwise Competitive Programming} \hfill UDLAP \\
\textit{Miembro} \hfill 2025 \\
Participación en competencias de programación algorítmica bajo presión de tiempo, con enfoque en la optimización de complejidad computacional. Desarrollo de soluciones eficientes en C++ y Python para problemas de grafos, estructuras de datos y programación dinámica. \\

\textbf{How Do Machines Learn?} \hfill UDLAP \\
\textit{Miembro} \hfill 2025 \\
Exploración aplicada del aprendizaje profundo utilizando librerías como PyTorch y CUDA. Implementación de redes neuronales desde cero y entrenamiento de modelos en GPU. Estudio de artículos y arquitecturas como CNN y RNN. \\




\begin{center}  
    \vspace{1ex}  
    \textbf{Habilidades} 
    \vspace{-1ex}  
\end{center}

\textbf{Lenguajes de Programación:} Python, R, C++, SQL, Java, Mosel, C.\\  
\textbf{Herramientas de Office:} Excel (VBA), Word, PowerPoint, Power BI.\\  
\textbf{Sistemas de Gestión de Bases de Datos:} MySQL.\\  
\textbf{Control de Versiones:} Git, GitHub.\\  
\textbf{Entornos de Scripting y Consola:} Bash (Arch Linux, Debian), PowerShell (Windows).\\
\textbf{Lenguajes de Marcado y Documentación:} HTML, LaTeX, Markdown.\\




\end{document}









